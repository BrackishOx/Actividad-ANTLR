\documentclass[12pt]{article}

\usepackage[spanish]{babel}
\usepackage[utf8]{inputenc}
\usepackage{listings}
\usepackage{xcolor}
\usepackage{geometry}
\geometry{a4paper, margin=2.5cm}

\title{Implementación de la Calculadora del Capítulo 4 en ANTLR}
\author{Santiago Céspedes}
\date{\today}

\definecolor{codegray}{rgb}{0.95,0.95,0.95}

\lstset{
    backgroundcolor=\color{codegray},
    basicstyle=\ttfamily\small,
    breaklines=true,
    frame=single
}

\begin{document}

\maketitle

\section{Objetivo}

Implementar una calculadora aritmética utilizando ANTLR 4.13.2 con Java como lenguaje objetivo, aplicando el patrón \textit{Visitor} explicado en el Capítulo 4. La calculadora permite evaluar expresiones con suma, resta, multiplicación, división, paréntesis y asignación de variables.

\section{Instalación y Configuración del Entorno}

El proyecto se desarrolló en WSL (Ubuntu en Windows).

\subsection*{Instalación de Java}

\begin{lstlisting}[language=bash]
sudo apt update
sudo apt install openjdk-17-jdk -y
\end{lstlisting}

\subsection*{Verificación}

\begin{lstlisting}[language=bash]
java -version
\end{lstlisting}

\subsection*{Instalación de ANTLR}

Se configuró el archivo \texttt{antlr-4.13.2-complete.jar} en el sistema y se creó el alias:

\begin{lstlisting}[language=bash]
alias antlr4='java -jar /usr/local/lib/antlr-4.13.2-complete.jar'
\end{lstlisting}

\section{Diseño de la Gramática}

Se creó el archivo \texttt{Calc.g4} con la siguiente estructura:

\begin{lstlisting}
grammar Calc;

prog:   stat+ ;

stat:   expr NEWLINE                # printExpr
    |   ID '=' expr NEWLINE         # assign
    |   NEWLINE                     # blank
    ;

expr:   expr op=('*'|'/') expr      # MulDiv
    |   expr op=('+'|'-') expr      # AddSub
    |   INT                         # int
    |   ID                          # id
    |   '(' expr ')'                # parens
    ;

MUL :   '*' ;
DIV :   '/' ;
ADD :   '+' ;
SUB :   '-' ;
ID  :   [a-zA-Z]+ ;
INT :   [0-9]+ ;
NEWLINE:'\r'? '\n' ;
WS  :   [ \t]+ -> skip ;
\end{lstlisting}

La gramática define:

\begin{itemize}
    \item Reglas sintácticas para expresiones aritméticas.
    \item Precedencia correcta de operadores.
    \item Tokens léxicos para números, identificadores y operadores.
\end{itemize}

\section{Generación del Parser}

Se generó el parser y visitor con el siguiente comando:

\begin{lstlisting}[language=bash]
antlr4 -visitor Calc.g4
\end{lstlisting}

Archivos generados:
\begin{itemize}
    \item CalcLexer.java
    \item CalcParser.java
    \item CalcBaseVisitor.java
    \item CalcVisitor.java
\end{itemize}

\section{Implementación del Visitor}

Se creó la clase \texttt{EvalVisitor.java} que extiende \texttt{CalcBaseVisitor<Integer>}.

El visitor evalúa recursivamente las expresiones y utiliza un \texttt{HashMap} para almacenar variables.

Ejemplo del método para suma y resta:

\begin{lstlisting}[language=Java]
@Override
public Integer visitAddSub(CalcParser.AddSubContext ctx) {
    int left = visit(ctx.expr(0));
    int right = visit(ctx.expr(1));
    if (ctx.op.getType() == CalcParser.ADD) 
        return left + right;
    return left - right;
}
\end{lstlisting}

\section{Clase Principal}

Se implementó la clase \texttt{Main.java} para ejecutar el parser:

\begin{lstlisting}[language=Java]
CharStream input = CharStreams.fromStream(System.in);
CalcLexer lexer = new CalcLexer(input);
CommonTokenStream tokens = new CommonTokenStream(lexer);
CalcParser parser = new CalcParser(tokens);

ParseTree tree = parser.prog();
EvalVisitor eval = new EvalVisitor();
eval.visit(tree);
\end{lstlisting}

\section{Compilación}

\begin{lstlisting}[language=bash]
javac -cp ".:/usr/local/lib/antlr-4.13.2-complete.jar" *.java
\end{lstlisting}

\section{Ejecución}

\begin{lstlisting}[language=bash]
java -cp ".:/usr/local/lib/antlr-4.13.2-complete.jar" Main
\end{lstlisting}

Ejemplo de ejecución:

\begin{lstlisting}
3+4*2
\end{lstlisting}

Salida:

\begin{lstlisting}
11
\end{lstlisting}

\section{Conclusiones}

Se logró implementar correctamente la calculadora utilizando ANTLR y el patrón Visitor en Java. El sistema respeta la precedencia de operadores y permite asignación de variables. La integración entre el lexer, parser y visitor demuestra el funcionamiento completo de un pequeño intérprete aritmético.

\end{document}
